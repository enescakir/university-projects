\documentclass[12pt,a4paper]{report}
\usepackage[utf8]{inputenc}
\usepackage{amsmath}
\usepackage{amsfonts}
\usepackage{amssymb}
\usepackage{fullpage}
\usepackage{graphicx}

\title{CMPE 240 Experiment 4 Preliminary Work}
\begin{document}

\noindent
\textbf{Student Names:} Ergün ERDOĞMUŞ, Mustafa Enes ÇAKIR \\
\textbf{Student IDs:} 2014400006, 2013400105 \\
\textbf{Group ID:} 2

\section*{CMPE 240 Experiment 4 Preliminary Work}

\subsection*{Step 1}

State the inputs and outputs of the state registers.

\begin{itemize}
	\item[$\Rightarrow$] Inputs: $n_2, n_1, n_0, $
	\item[$\Rightarrow$] Outputs: $s_2, s_1, s_0, $
\end{itemize}

\subsection*{Step 2}

State the inputs and outputs of the combinational block of the sequential circuit.

\begin{itemize}
	\item[$\Rightarrow$] Inputs: $s_2, s_1, s_0, x $
	\item[$\Rightarrow$] Outputs: $n_2, n_1, n_0, y_1, y_0$
\end{itemize}

\subsection*{Step 3}

Write each output (including next state bits) as a function of the inputs.
\\
$n_2$ = $x s_2 s_1's_0' + x s_2' s_1 s_0$\\
$n_1$ = $x s_2's_1' s_0 + x s_2' s_1 s_0'$ \\
$n_0$ = $x' s_2 s_1' s_0' + x' s_2' + x s_2' s_1 s_0'$\\
$y_1$ = $x' s_2 s_1' s_0' + x' s_2' s_1 s_0$\\
$y_0$ = $x' s_2 s_1' s_0' + x' s_2' s_1 s_0'$\\

\pagebreak
\noindent
\textbf{Student Names:} Ergün ERDOĞMUŞ, Mustafa Enes ÇAKIR \\
\textbf{Student IDs:} 2014400006, 2013400105 \\
\textbf{Group ID:} 2

\subsection*{Step 4}

Draw the truth table for the combinational circuit.
\begin{center}
	\begin{tabular}{|c|cccc|ccccc|}
		\hline
		\textbf{\#} & \textbf{S2} & \textbf{S1} & \textbf{S0} & \textbf{X} & \textbf{N2} & \textbf{N1} & \textbf{N0} & \textbf{Y1} & \textbf{Y0} \\
		\hline
		0 & 0 & 0 & 0 & 0 & 0 & 0 & 1 & 0 & 0 \\
		\hline
		1 & 0 & 0 & 0 & 1 & 0 & 0 & 0 & 0 & 0 \\
		\hline
		2 & 0 & 0 & 1 & 0 & 0 & 0 & 1 & 0 & 0 \\
		\hline
		3 & 0 & 0 & 1 & 1 & 0 & 1 & 0 & 0 & 0 \\
		\hline
		4 & 0 & 1 & 0 & 0 & 0 & 0 & 1 & 0 & 1 \\
		\hline
		5 & 0 & 1 & 0 & 1 & 0 & 1 & 1 & 0 & 0 \\
		\hline
		6 & 0 & 1 & 1 & 0 & 0 & 0 & 1 & 1 & 0 \\
		\hline
		7 & 0 & 1 & 1 & 1 & 1 & 0 & 0 & 0 & 0 \\
		\hline
		8 & 1 & 0 & 0 & 0 & 0 & 0 & 1 & 1 & 1 \\
		\hline
		9 & 1 & 0 & 0 & 1 & 1 & 0 & 0 & 0 & 0 \\
		\hline
		10 & 1 & 0 & 1 & 0 & 0 & 0 & 0 & 0 & 0 \\
		\hline
		11 & 1 & 0 & 1 & 1 & 0 & 0 & 0 & 0 & 0 \\
		\hline
		12 & 1 & 1 & 0 & 0 & 0 & 0 & 0 & 0 & 0 \\
		\hline
		13 & 1 & 1 & 0 & 1 & 0 & 0 & 0 & 0 & 0 \\
		\hline
		14 & 1 & 1 & 1 & 0 & 0 & 0 & 0 & 0 & 0 \\
		\hline
		15 & 1 & 1 & 1 & 1 & 0 & 0 & 0 & 0 & 0 \\
		\hline
	\end{tabular}
\end{center}

\subsection*{Step 5}
Draw the finite state machine by using the truth table.
\begin{center}
	\includegraphics[width=400pt,keepaspectratio]{exp4-fsm.png}
\end{center}

\pagebreak
\noindent
\textbf{Student Names:} Ergün ERDOĞMUŞ, Mustafa Enes ÇAKIR \\
\textbf{Student IDs:} 2014400006, 2013400105 \\
\textbf{Group ID:} 2

\subsection*{Step 6}
How many unreachable states does the finite state machine contain? (No explanation, only short answer).
\begin{itemize}
	\item[$\Rightarrow$] \# of Unreachable States: 3
\end{itemize}

\subsection*{Step 7}

Briefly explain the relation between the input and the output.

\begin{itemize}
	\item[$\Rightarrow$] Explanation: In this FSM, output depends not solely on the state but also on the given input. What it does is this: after a start with 0 it counts the number of consecutive 1's up to 3 with start and end with 0 (like 0110). If the count is 0, outputs 0; if the count is 1, output is 01; if the count is 2, output is 10 and if the count is 3, output is 11. If the count is greater than 1, it also outputs 11.
\end{itemize}

\end{document}
